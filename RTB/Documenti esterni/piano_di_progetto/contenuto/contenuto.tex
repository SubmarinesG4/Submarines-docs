%Contenuto del documento
%Introduzione
\section{Introduzione}
\subsection{Scopo del Documento}
Lo scopo di questo documento è fornire un prospetto dettagliato riguardare la pianificazione e le modalità tramite le quali verrà sviluppato il progetto.\\
Il documento tratterà, in ordine:
\begin{itemize}
    \item L'Analisi dei Rischi,
    \item La descrizione del modello di sviluppo adottato,
    \item La suddivisione delle fasi e l'assegnazione dei ruoli
    \item La stima dei costi e delle risorse necessarie allo sviluppo.
\end{itemize}

\subsection{Scopo del Prodotto}
Lo scopo di (NOME-PROGETTO) e di Zero12 è la creazione di una piattaforma in grado di gestire i testi delle localizzazioni di mobile apps e webapps. \\
Il sistema, gestito in modalità multi-tenant, sarà costituito principalmente da un'API\glo{} che permette agli sviluppatori di accedere alle traduzioni dei loro 
testi da inserire all'interno delle apps, e da una webapp di backoffice (CMS) che permette agli amministratori del sistema di accedere al database di traduzioni.

\subsection{Glossario}
\gloDesc

\subsection{Riferimenti}
\subsubsection{Riferimenti Normativi} % DA RIVEDERE
\begin{itemize}
    \item \emph{NormeDiProgetto-v1.0.0};
    \item Regolamento del progetto didattico: \\ \href{https://www.math.unipd.it/~tullio/IS-1/2022/Dispense/PD02.pdf}{\color{blue}https://www.math.unipd.it/~tullio/IS-1/2022/Dispense/PD02.pdf}
    \item Capitolato d'appalto C4: \\ \href{https://www.math.unipd.it/~tullio/IS-1/2022/Progetto/C4.pdf}{\color{blue}https://www.math.unipd.it/~tullio/IS-1/2022/Progetto/C4.pdf}
\end{itemize}

\subsubsection{Riferimenti Informativi} % DA RIVEDERE
\begin{itemize}
    \item \emph{PianoDiQualifica-v1.0.0};
    \item I processi del ciclo di vita del software - Materiale didattico del corso IdS \\ \href{https://www.math.unipd.it/~tullio/IS-1/2022/Dispense/T02.pdf}{\color{blue}https://www.math.unipd.it/~tullio/IS-1/2022/Dispense/T02.pdf}
    \item Gestione di progetto - Materiale didattico del corso IdS \\ \href{https://www.math.unipd.it/~tullio/IS-1/2022/Dispense/T04.pdf}{\color{blue}https://www.math.unipd.it/~tullio/IS-1/2022/Dispense/T04.pdf}
\end{itemize}

\subsection{Scadenze}
Dopo opportune valutazione, il gruppo \emph{Submarines} si impegna a rispettare le seguenti scadenze per lo svolgimento del progetto (NOME-PROGETTO):
\begin{itemize}
    \item \textbf{\RTB{} (RTB)}: settimana dal (DATA) a (DATA);
    \item \textbf{\PB{} (PB)}: settimana dal (DATA) a (DATA);
    \item \textbf{\CA{} (CA)}: settimana dal (DATA) a (DATA);
\end{itemize}

%Analisi dei Rischi
\section{Analisi dei rischi}
L'Analisi dei Rischi è un processo tramite il quale si cerca di prevedere e valutare gli eventuali rischi in cui si può
incorrere durante lo sviluppo di un progetto. La procedura per la gestione di tali rischi può essere suddiviso in 4 attività:
\begin{itemize}
    \item \textbf{Identificazione dei rischi}: Individuazione di eventuali problematiche che possono compromettere l'avanzamento;
    \item \textbf{Analisi dei rischi}: Individuazione delle conseguenze del rischio sul progetto e della probabilità di occorrenza;
    \item \textbf{Attività 3}: Descrizione
    \item \textbf{Attività 4}: Descrizione
\end{itemize}

\subsection{Rischi tecnologici}
Test

\subsection{Rischi personali}
Test

\subsection{Rischi organizzativi}
Test

\subsection{Rischi legati ai requisiti}
Test

%Modello di sviluppo
\section{Modello di sviluppo}
\subsection{Modello incrementale} %Il modello è DA SCEGLIERE
Test

\section{Pianificazione}
