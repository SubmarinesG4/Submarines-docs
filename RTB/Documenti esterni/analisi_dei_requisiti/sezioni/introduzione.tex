\section{Introduzione}
\subsection{Glossario}
Per chiarezza c'è un documento "Glossario v.1.0.0" presente nella documentazione che va a chiarire tutti i termini e le espressioni che possono risultare ambigue. Questi termini avranno il pedice 'G', esempio "parola\textsubscript G".
In caso venga citato un documento verrà inserito il pedice 'D', esempio "documento\textsubscript D".
\subsection{Riferimenti}
\subsubsection{Riferimenti normativi}
\begin{itemize}
    \item Capitolato C4: \url{https://www.math.unipd.it/~tullio/IS-1/2022/Progetto/C4.pdf}
    \item Verbale esterno 2022-10-25.pdf\textsubscript D: \url{https://github.com/SubmarinesG4/SubmarinesG4-SWE/blob/main/RTB/Documenti%20esterni/Verbali/Verbale%20esterno%202022-10-25.pdf}
    \item Norme di progetto v.0.0.1\textsubscript D: \url{INSERIRE URL}
\end{itemize}

\subsubsection{Riferimenti informativi}
\begin{itemize}
    \item Slides lezione T6 - Analisi dei requisiti: \url{https://www.math.unipd.it/~tullio/IS-1/2022/Dispense/T06.pdf}
    \item Slides lezione    - Analisi e descrizione delle funzionalità: use case e relativi diagrammi (UML): \url{https://www.math.unipd.it/~rcardin/swea/2022/Diagrammi%20Use%20Case.pdf}
\end{itemize}