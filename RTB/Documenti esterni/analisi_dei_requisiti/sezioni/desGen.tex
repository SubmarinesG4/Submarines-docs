\section{Descrizione generale del prodotto}
\subsection{Scopo del prodotto}
Lo scopo di *INSERIRE NOME SOFTWARE* è di essere un insieme di API che forniscano traduzioni multi-lingua. Saranno utilizzate quindi da terzi per effetuare traduzioni sulle applicazioni di loro proprietà.
\subsection{Target}
Il target del software saranno aziende proprietarie di applicativi che necessitano di un meccanismo di traduzione per interfacciarsi al mercato internazionale.
\subsection{Attori}
    \subsubsection{Attori principali}
        \paragraph{Utente traduttore}
        Sono gli utenti base del sistema. A loro è adibito il compito di creare, modificare, approvare o eliminare le traduzioni. Sono la vera e propria "forza-lavoro".
        \paragraph{Admin}
        Sono gli utenti che dovranno gestire il tenant a loro assegnato. A loro è permesso di fare qualsiasi operazione sulle traduzioni. Si differenziano dagli utenti traduttori nel fatto che possono effettuare la pubblicazione delle traduzioni.
        \paragraph{SuperAdmin}
        Sono gli utenti con più privilegi di tutto il sistema. Rappresenta una persona dell'azienda Zero12 che andrà a gestire i vari tenant (clienti) e le loro caratteristiche. Hanno la possibilità di eseguire tutte le operazioni che gli altri utenti hanno.
    \subsubsection{Attori secondari}
\section{Obblighi progettuali}