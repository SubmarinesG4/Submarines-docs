\section{Casi d'uso}
\subsection{Scopo}
In questa sezione verranno presentati gli use cases tracciati, che il software dovrà implementare.
\subsection*{UC 1 - Accesso}
    \subsubsection*{UC 1.1 - Prima autenticazione}
    Flusso di eventi: UC1.1.1
        \paragraph*{UC 1.1.1 - Cambio password obbligatorio}
        \paragraph*{UC 1.1.2 - Inserimento indirizzo email obbligatorio}
        \paragraph*{UC 1.1.3 - Inserire dati personali}
    \subsubsection*{UC 1.2 - Autenticazione generica}
        \paragraph*{UC 1.2.1 - Inserimento indirizzo email}
        \paragraph*{UC 1.2.2 - Inserimento password}
    \subsubsection*{UC 1.3 - Password dimenticata}
    Flusso di eventi: UC1.3.1 UC1.3.2 UC1.3.3
        \paragraph*{UC 1.3.1 - Inserimento indirizzo email per recupero}
        \paragraph*{UC 1.3.2 - Inserimento nuova password}
        \paragraph*{UC 1.3.3 - Conferma password}
\subsection*{UC 2 - Logout} %! Mettere area personale come user case?
\subsection*{UC 3 - Cambio password}
    \begin{itemize}
        \item Attore primario: utente autenticato qualsiasi.
        \item Precondizione: l'utente vuole cambiare la propria password.
        \item Postcondizione: l'utente ha cambiato password.
        \item Scenario principale: la password dell'utente in seguito a eventi esterni non è più sicura.
        \item Flusso di eventi:
            \begin{enumerate}
                \item L'utente inserisce la vecchia password per riconfermare la sua identità [UC 3.1].
                \item L'utente inserisce la nuova password [UC 3.2].
                \item Per sicurezza il sistema chiede il reinserimento della nuova password [UC 3.3].
            \end{enumerate}
    \end{itemize}
    \subsubsection*{UC 3.1 - Inserimento password vecchia}
        \begin{itemize}
            \item Attore primario: utente autenticato qualsiasi.
            \item Precondizione: l'utente ha necessità di cambiare password.
            \item Postcondizione: l'utente ha inserito la vecchia password.
            \item Scenario principale: inserimento vecchia password per modificarla.
            \item Estensione: se la password è sbagliata quando viene premuto il tasto "Conferma" viene visualizzato un messaggio d'errore e non viene effettuato il cambio password.
        \end{itemize}
    \subsubsection*{UC 3.2 - Inserimento password nuova}
        \begin{itemize}
            \item Attore primario: utente autenticato qualsiasi.
            \item Precondizione: l'utente ha necessità di cambiare password ed ha inserito la vecchia password.
            \item Postcondizione: l'utente non autenticato ha inserito la nuova password.
            \item Scenario principale: inserimento nuova password.
            \item Estensione: se la nuova password inserita non rispetta gli standard necessari viene visualizzato un messaggio d'errore e non viene effettuato il cambio password.
        \end{itemize}
    \subsubsection*{UC 3.3 - Conferma password}
        \begin{itemize}
            \item Attore primario: utente autenticato qualsiasi.
            \item Precondizione: l'utente ha necessità di cambiare password ed ha inserito la vecchia e la nuova password.
            \item Postcondizione: l'utente non autenticato ha inserito di nuovo la nuova password.
            \item Scenario principale: conferma nuova password per evitare errori.
            \item Estensione: se la password non corrisponde a quella inserita sopra viene visualizzato un messaggio d'errore e non viene effettuato il cambio password.
        \end{itemize}
\subsection*{UC 4 - Inserimento traduzione}
    \begin{itemize}
    \item Attore primario: utente autenticato qualsiasi.
    \item Precondizione: l'utente vuole inserire una nuova traduzione.
    \item Postcondizione: l'utente ha correttamente inserito una nuova traduzione.
    \item Scenario principale: l'utente clicca sul pulsante "Nuova traduzione", compila i campi dati relativi e clicca il pulsante conferma.
    \item Flusso di eventi:
        \begin{enumerate}
            \item L'utente clicca sul pulsante "Nuova traduzione".
            \item L'utente inserisce la lingua da tradurre [UC 4.1].
            \item L'utente inserisce la lingua in cui bisogna tradurre [UC 4.2].
            \item L'utente inserisce l'espressione da traddure [UC 4.3].
            \item L'utente inserisce l'espressione tradotta [UC 4.4].
        \end{enumerate}
    \end{itemize}
    \subsubsection*{UC 4.1 - Inserimento lingua da tradurre}
        \begin{itemize}
            \item Attore primario: utente autenticato qualsiasi.
            \item Precondizione: l'utente ha necessità di inserire una traduzione.
            \item Postcondizione: l'utente ha inserito la lingua da tradurre.
            \item Estensione: l'utente deve inserire obbligatoriamente una lingua da tradurre altrimenti viene visualizzato un messaggio d'errore e non viene salvata la traduzione.
        \end{itemize}
    \subsubsection*{UC 4.2 - Inserimento lingua tradotta}
        \begin{itemize}
            \item Attore primario: utente autenticato qualsiasi.
            \item Precondizione: l'utente ha necessità di inserire una traduzione.
            \item Postcondizione: l'utente ha inserito la lingua in cui tradurre.
            \item Estensione: l'utente deve inserire obbligatoriamente una lingua in cui tradurre altrimenti viene visualizzato un messaggio d'errore e non viene salvata la traduzione.
        \end{itemize}
    \subsubsection*{UC 4.3 - Inserimento espressione lingua da tradurre}
        \begin{itemize}
            \item Attore primario: utente autenticato qualsiasi.
            \item Precondizione: l'utente ha necessità di inserire una traduzione.
            \item Postcondizione: l'utente ha inserito l'espressione da tradurre.
            \item Estensione: l'utente deve inserire obbligatoriamente almeno un carattere altrimenti viene visualizzato un messaggio d'errore e non viene salvata la traduzione.
        \end{itemize}
    \subsubsection*{UC 4.4 - Inserimento espressione lingua tradotta}
        \begin{itemize}
            \item Attore primario: utente autenticato qualsiasi.
            \item Precondizione: l'utente ha necessità di inserire una traduzione.
            \item Postcondizione: l'utente ha inserito l'espressione in cui tradurre.
            \item Estensione: l'utente deve inserire obbligatoriamente almeno un carattere altrimenti viene visualizzato un messaggio d'errore e non viene salvata la traduzione.
        \end{itemize}
\subsection*{UC 5 - Revisione traduzione}
    \subsubsection*{UC 5.1 - Approvazione}
    \subsubsection*{UC 5.2 - Eliminazione}
    Flusso di eventi: utente schiaccia elimina, utente schiaccia conferma elimina
\subsection*{UC 6 - Modifica traduzione}
Flusso di eventi: UC6.1 UC6.2 UC6.3 UC6.4   %! Eventualmente togliere il modifica lingue
    \subsubsection*{UC 6.1 - Modifica lingua da tradurre}
    \subsubsection*{UC 6.2 - Modifica lingua tradotta}
    \subsubsection*{UC 6.3 - Modifica espressione lingua da tradurre}
    \subsubsection*{UC 6.4 - Modifica espressione lingua tradotta}
\subsection*{UC 7 - Elimina traduzione} %! Eliminazione di più traduzioni?
Flusso di eventi: utente schiaccia elimina, utente schiaccia conferma elimina
\subsection*{UC 8 - Ricerca}
    \subsubsection*{UC 8.1 - Ricerca tramite parola}
        \paragraph*{UC 8.1.1 - Inserimento parola}
    \subsubsection*{UC 8.2 - Ricerca tramite id univoco}
        \paragraph*{UC 8.2.1 - Inserimento id}
\subsubsection*{UC 9 - Filtraggio ricerca}
    \subsubsection*{UC 9.1 - Filtraggio per lingua da tradurre}
    \subsubsection*{UC 9.2 - Filtraggio per lingua tradotta}
    \subsubsection*{UC 9.3 - Filtraggio per data di creazione}
    \subsubsection*{UC 9.4 - Filtraggio per utente traduttore}
    \subsubsection*{UC 9.5 - Filtraggio per approvazione}
    \subsubsection*{UC 9.6 - Filtraggio per pubblicazione}
\subsection*{UC 10 - Pubblicazione di una traduzione} %! Pubblicazione di più traduzioni?
\subsection*{UC 11 - Aggiunta cliente}
    \subsubsection*{UC 11.1 - Inserimento nome cliente}
    \subsubsection*{UC 11.2 - Inserimento numero traduzioni disponibili}
\subsection*{UC 12 - Eliminazione cliente}
